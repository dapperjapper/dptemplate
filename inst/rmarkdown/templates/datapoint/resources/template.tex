\documentclass[12pt,letterpaper,twoside]{article}
\usepackage[utf8]{inputenc}
\RequirePackage{graphicx,color,ae,fancyvrb, hyperref}
\RequirePackage[T1]{fontenc}
\usepackage{framed}
\usepackage[utf8]{inputenc}
\usepackage{wallpaper}
\usepackage[none]{hyphenat}
\usepackage{fancyhdr}
\usepackage[letterpaper,left=1in,right=1in,top=1in,bottom=1.25in,footskip=.6in]{geometry}
\usepackage{enumitem}
\usepackage{amssymb}
\usepackage[geometry]{ifsym}
\usepackage{tocbibind}
\usepackage{tocloft}
\usepackage{sectsty}
\usepackage{booktabs}

% I will use biblatex for the bibliography
\usepackage[
backend=bibtex,
style=authoryear,
bibencoding=ascii,
minnames=1,
maxnames=4,
dashed=false,
firstinits=false,
sorting=nyt
]{biblatex}

$if(bibliography)$
\addbibresource{$bibliography$}
$endif$

% This block comes from Rob Hyndman's blog
\DeclareFieldFormat{url}{\url{#1}}
\DeclareFieldFormat[article]{pages}{#1}
\DeclareFieldFormat[inproceedings]{pages}{\lowercase{pp.}#1}
\DeclareFieldFormat[incollection]{pages}{\lowercase{pp.}#1}
\DeclareFieldFormat[article]{volume}{\mkbibbold{#1}}
\DeclareFieldFormat[article]{number}{\mkbibparens{#1}}
\DeclareFieldFormat[article]{title}{\MakeCapital{#1}}
\DeclareFieldFormat[article]{url}{}
\DeclareFieldFormat[book]{url}{}
\DeclareFieldFormat[inbook]{url}{}
\DeclareFieldFormat[incollection]{url}{}
\DeclareFieldFormat[inproceedings]{url}{}
\DeclareFieldFormat[inproceedings]{title}{#1}
\DeclareFieldFormat{shorthandwidth}{#1}
\renewbibmacro{in:}{%   Removes In: for an article
  \ifentrytype{article}{}{%
  \printtext{\bibstring{in}\intitlepunct}}}
\AtEveryBibitem{\clearfield{month}}
\AtEveryCitekey{\clearfield{month}}

%% Sweave(-like)
%% These commands were taken from the rticles package
\DefineVerbatimEnvironment{Sinput}{Verbatim}{fontshape=sl}
\DefineVerbatimEnvironment{Soutput}{Verbatim}{}
\DefineVerbatimEnvironment{Scode}{Verbatim}{fontshape=sl}
\newenvironment{Schunk}{}{}
\DefineVerbatimEnvironment{Code}{Verbatim}{}
\DefineVerbatimEnvironment{CodeInput}{Verbatim}{fontshape=sl}
\DefineVerbatimEnvironment{CodeOutput}{Verbatim}{}
\newenvironment{CodeChunk}{}{}
\setkeys{Gin}{width=0.8\textwidth}

%% paragraphs
%%\setlength{\parskip}{0.7ex plus0.1ex minus0.1ex}
\setlength{\parskip}{12pt}
\setlength{\parindent}{0em}

\def\maketitle{
  {\color{white} \fontfamily{phv} \fontsize{38}{38} \selectfont CFPB Data Point:\\[25pt]
  $dp_title$}\\[40pt]
  {\noindent \color{white} \fontfamily{phv} \fontsize{14}{14} \selectfont The CFPB Office of Research\par}
  \newpage
  }

\addtolength{\wpXoffset}{-2in}
\addtolength{\wpYoffset}{-4.7in}

\renewcommand{\contentsname}{Table of contents}
\renewcommand\cftsecleader{\cftdotfill{\cftdotsep}}
\renewcommand{\cftsecaftersnum}{.}%

\sectionfont{\mdseries\fontfamily{phv} \fontsize{32}{32} \selectfont}
\makeatletter
\renewcommand{\@seccntformat}[1]{\csname the#1\endcsname.\quad}
\makeatother

% Start each section on a new page
\let\stdsection\section
\renewcommand\section{\newpage\vspace*{1.5in}\stdsection}

\providecommand{\tightlist}{%
  \setlength{\itemsep}{0pt}\setlength{\parskip}{0pt}}

$preamble$

\fancyhf{}
\renewcommand{\headrulewidth}{0pt}
\pagestyle{fancy}

% This is beta on the TOC
\renewcommand{\cfttoctitlefont}{\fontfamily{phv} \fontsize{32}{32} \selectfont}

% This changes the spacing of text lines
%\renewcommand{\baselinestretch}{1.4}
\usepackage{setspace}
\onehalfspacing

% Adjust the placement and format of captions
\usepackage{textcase}
\usepackage[figureposition=top,figurename=FIGURE,
            tableposition=top,tablename=TABLE]{caption}
\DeclareCaptionTextFormat{up}{\MakeTextUppercase{#1}}
\DeclareCaptionFont{ninept}{\fontsize{9pt}{11pt}\selectfont #1}
\captionsetup[figure]{
  font=ninept,
  format=hang,
  labelsep=colon,
  justification=raggedright,
  singlelinecheck=false,
  textformat=up,
  labelfont=bf,
  textfont=up}
\captionsetup[table]{
  font=ninept,
  format=hang,
  labelsep=colon,
  justification=raggedright,
  singlelinecheck=false,
  textformat=up,
  labelfont=bf,
  textfont=up}

\usepackage{floatrow}
\floatsetup[figure]{capposition=top}
\floatsetup[table]{capposition=top}

% This adjusts the footnotes
\usepackage[hang,flushmargin]{footmisc}
\setlength{\footnotesep}{2em}

% This adds row coloring to the tables
\usepackage[table]{xcolor}
\definecolor{greenhead}{cmyk}{0.70, 0, 0.89, 0.01}
\definecolor{green1}{cmyk}{0.20, 0, 0.28, 0}
\definecolor{green2}{cmyk}{0.30, 0.01, 0.48, 0}

\begin{document}
\rowcolors{2}{green2}{green1}

% Cover Page
\begin{titlepage}
  \thispagestyle{fancy}
  \ThisULCornerWallPaper{1}{dataPointCoverBackground}
  \ThisCenterWallPaper{0.35}{cfpblogo_wide}
  \fancyfoot[R]{\fontfamily{phv} \fontsize{10}{10} \selectfont $dp_date$}
  \vspace*{1.5in}
  {\color{white} \fontfamily{phv} \fontsize{38}{38} \selectfont CFPB Data Point:\\[25pt]
  $dp_title$\\[40pt]
  \fontsize{14}{14} \selectfont The CFPB Office of Research\par}
\end{titlepage}

% Author Page
\raggedright
\fancyfoot[R]{}
\fancyfoot[L]{\fontfamily{phv} \fontsize{8}{8} \selectfont \thepage\hspace{20pt}CFPB DATA POINT: \uppercase{$dp_title$}}
\begin{itemize}     % [label=\FilledSmallSquare]
 \renewcommand{\labelitemi}{\scriptsize\FilledSmallSquare}
  $for(dp_author)$
    \item $dp_author$
  $endfor$
\end{itemize}

This is another in an occasional series of publications from the Consumer Financial Protection Bureau's Office of Research. These publications are intended to further the Bureau's objective of providing an evidence-based perspective on consumer financial markets, consumer behavior, and regulations to inform the public discourse.
\newpage

\vspace*{1.5in}
\tableofcontents
\thispagestyle{fancy}
% \clearpage
% {
%   \pagestyle{empty}
%   \fancyfoot[L]{\fontfamily{phv} \fontsize{8}{8} \selectfont \thepage\hspace{20pt}CFPB DATA POINT: \uppercase{$dp_title$}}
% \vspace*{1.5in}
% \tableofcontents
% \thispagestyle{empty}
% }

\newpage


% Body of the report
$body$

$if(bibliography)$
\newpage
\printbibliography[heading=bibintoc,title={References}]
$endif$

%$if(natbib)$
%$if(biblio-files)$
%$if(biblio-title)$
%$if(book-class)$
%\renewcommand\bibname{$biblio-title$}
%$else$
%\renewcommand\refname{$biblio-title$}
%$endif$
%$endif$
%\newpage
%\vspace*{1.5in}
%{\fontfamily{phv} \fontsize{32}{32} \selectfont References}
%\bibliography{$biblio-files$}
%$endif$
%$endif$

%$if(biblatex)$
%\newpage
%\vspace*{1.5in}
%{\fontfamily{phv} \fontsize{32}{32} \selectfont References}
%\printbibliography
%$if(biblio-title)$[title=$biblio-title$]$endif$
%$endif$

\end{document}
