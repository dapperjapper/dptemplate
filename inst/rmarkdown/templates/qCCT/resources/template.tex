\documentclass[12pt,letterpaper,twoside]{article}
\usepackage[utf8]{inputenc}
\RequirePackage{graphicx,color,ae,fancyvrb, hyperref}
\RequirePackage[T1]{fontenc}
\usepackage{framed}
\usepackage[utf8]{inputenc}
\usepackage{wallpaper}
\usepackage[none]{hyphenat}
\usepackage{fancyhdr}
\usepackage[letterpaper,left=1in,right=1in,top=1.2in,bottom=1.25in,footskip=.6in]{geometry}
\usepackage{enumitem}
\usepackage{amssymb}
\usepackage[geometry]{ifsym}
\usepackage{tocbibind}
\usepackage{tocloft}
\usepackage{sectsty}
\usepackage{booktabs}
\usepackage[all]{nowidow}
\usepackage{tgschola}
\usepackage{lscape}

%\usepackage{fontspec}
%\usepackage{etex}

\renewcommand{\arraystretch}{1.5}

%\setlist{noitemsep}

$if(draft)$
\usepackage{draftwatermark}
$endif$

% I will use biblatex for the bibliography
\usepackage[
backend=bibtex,
style=authoryear,
bibencoding=ascii,
minnames=1,
maxnames=4,
dashed=false,
firstinits=false,
sorting=nyt
]{biblatex}

$if(bibliography)$
\addbibresource{$bibliography$}
$endif$

% This block comes from Rob Hyndman's blog
\DeclareFieldFormat{url}{Available online at {\color{hyperblue} \url{#1}}}
\DeclareFieldFormat[article]{pages}{#1}
\DeclareFieldFormat[inproceedings]{pages}{\lowercase{pp.}#1}
\DeclareFieldFormat[incollection]{pages}{\lowercase{pp.}#1}
%\DeclareFieldFormat[article]{volume}{\mkbibbold{#1}}
%\DeclareFieldFormat[article]{number}{\mkbibparens{#1}}
\DeclareFieldFormat[article]{title}{\MakeCapital{#1}}
\DeclareFieldFormat[article]{url}{}
\DeclareFieldFormat[book]{url}{}
\DeclareFieldFormat[inbook]{url}{}
\DeclareFieldFormat[incollection]{url}{}
\DeclareFieldFormat[inproceedings]{url}{}
\DeclareFieldFormat[techreport]{url}{}
\DeclareFieldFormat[inproceedings]{title}{#1}
\DeclareFieldFormat{shorthandwidth}{#1}
\renewbibmacro{in:}{%   Removes In: for an article
  \ifentrytype{article}{}{%
  \printtext{\bibstring{in}\intitlepunct}}}
%\AtEveryBibitem{\clearfield{month}}
\AtEveryCitekey{\clearfield{month}}

%% Sweave(-like)
%% These commands were taken from the rticles package
\DefineVerbatimEnvironment{Sinput}{Verbatim}{fontshape=sl}
\DefineVerbatimEnvironment{Soutput}{Verbatim}{}
\DefineVerbatimEnvironment{Scode}{Verbatim}{fontshape=sl}
\newenvironment{Schunk}{}{}
\DefineVerbatimEnvironment{Code}{Verbatim}{}
\DefineVerbatimEnvironment{CodeInput}{Verbatim}{fontshape=sl}
\DefineVerbatimEnvironment{CodeOutput}{Verbatim}{}
\newenvironment{CodeChunk}{}{}
\setkeys{Gin}{width=0.8\textwidth}

%% paragraphs
%%\setlength{\parskip}{0.7ex plus0.1ex minus0.1ex}
\setlength{\parskip}{12pt}
\setlength{\parindent}{0em}

% I don't think I need this anymore
%\def\maketitle{
%  {\color{white} \fontfamily{phv} \fontsize{38}{38} \selectfont CFPB Data Point:\\[25pt]
%  $dp_title$}\\[40pt]
%  {\noindent \color{white} \fontfamily{phv} \fontsize{14}{14} \selectfont The CFPB Office of %Research\par}
%  \newpage
%  }

% This controls where the CFPB logo appears on the cover sheet
% Positions are relative to the page center
\addtolength{\wpXoffset}{-2.3in}
\addtolength{\wpYoffset}{-3.8in}

\renewcommand{\contentsname}{Table of Contents}
\renewcommand{\cftsecfont}{\bfseries\fontfamily{phv} \fontsize{12}{12} \selectfont}
\renewcommand{\cftsecleader}{\bfseries\cftdotfill{\cftdotsep}}
\renewcommand{\cftsecpagefont}{\bfseries\fontfamily{phv} \fontsize{12}{12} \selectfont}
\renewcommand{\cftsecaftersnum}{.}%
\renewcommand{\cftsecpresnum}{\bfseries\color{black}}
\renewcommand{\cftsubsecpresnum}{\bfseries\color{black}}
\renewcommand{\cftsubsubsecpresnum}{\bfseries\color{black}}
\renewcommand{\cftsubsecafterpnum}{\hspace*{0.5in}}

\sectionfont{\bfseries\fontfamily{phv} \fontsize{14}{14} \selectfont}
\subsectionfont{\mdseries\fontfamily{phv} \fontsize{12}{12} \selectfont}
\subsubsectionfont{\mdseries\fontfamily{phv} \fontsize{12}{12} \selectfont}
\makeatletter
%\renewcommand{\@seccntformat}[1]{\csname the#1\endcsname.\quad}
%\renewcommand{\@seccntformat}[1]{\csname the#1\endcsname.\hspace*{0.5em}}

\DeclareRobustCommand{\@seccntformat}[1]{%
  \def\temp@@a{#1}%
  \def\temp@@b{section}%
  \ifx\temp@@a\temp@@b
  \csname \endcsname
  %\csname the#1\endcsname.\hspace*{0.25em}%
  \else
  \csname \endcsname
  %\csname the#1\endcsname\hspace*{0.25em}%
  \fi
}
\makeatother

% Start each section on a new page
%\let\stdsection\section
%\renewcommand\section{\newpage\vspace*{1.5in}\stdsection}

\providecommand{\tightlist}{%
  \setlength{\parskip}{0pt}\setlength{\topsep}{0pt}}

$preamble$

\fancyhf{}
\renewcommand{\headrulewidth}{0pt}
\pagestyle{fancy}

% This is beta on the TOC
\renewcommand{\cfttoctitlefont}{\fontfamily{phv} \fontsize{32}{32} \selectfont}

% This changes the spacing of text lines
%\renewcommand{\baselinestretch}{1.4}
\usepackage{setspace}
\onehalfspacing

% Adjust the placement and format of captions
\usepackage{textcase}
\usepackage[figureposition=top,figurename=FIGURE,
            tableposition=top,tablename=TABLE]{caption}
\DeclareCaptionTextFormat{up}{\MakeTextUppercase{#1}}
\DeclareCaptionFont{ninept}{\fontsize{9pt}{11pt}\selectfont #1}
%\DeclareCaptionLabelFormat{mylabel}{#1 #2.\hspace{5em}}
\DeclareCaptionLabelSeparator{mysep}{:\hspace*{0.2in}}
\captionsetup[figure]{
  font=ninept,
  format=hang,
  labelsep=mysep, % colon,
  justification=raggedright,
  singlelinecheck=false,
  textformat=up,
  labelfont=bf,
  textfont=up}
\captionsetup[table]{
  font=ninept,
  format=hang,
  labelsep=mysep,  % colon,
  justification=raggedright,
  singlelinecheck=false,
  textformat=up,
  labelfont=bf,
  textfont=up}

\usepackage{floatrow}
\floatsetup[figure]{capposition=top}
\floatsetup[table]{capposition=top}

% This adjusts the footnotes
\usepackage[hang,flushmargin]{footmisc}
%\setlength{\footnotesep}{2em}
\setlength{\footnotemargin}{0.1in}

% This adds row coloring to the tables
\usepackage[table]{xcolor}
% These are the green colors for the original table format
%\definecolor{greenhead}{cmyk}{0.70, 0, 0.89, 0.01}
%\definecolor{green1}{cmyk}{0.20, 0, 0.28, 0}
%\definecolor{green2}{cmyk}{0.30, 0.01, 0.48, 0}
\definecolor{greenhead}{cmyk}{0.26, 0.01, 0.40, 0}
\definecolor{green1}{cmyk}{0.1, 0.02, 0.12, 0}
\definecolor{green2}{cmyk}{0, 0, 0, 0}
\definecolor{hyperblue}{cmyk}{0.73, 0.35, 0, 0}


\begin{document}
\rowcolors{2}{green2}{green1}

% Cover Page
\begin{titlepage}
  % This is the old cover page (green cover)
  %\thispagestyle{fancy}
  %\ThisULCornerWallPaper{1}{dataPointCoverBackground}
  %\ThisCenterWallPaper{0.35}{cfpblogo_wide}
  %\fancyfoot[R]{\fontfamily{phv} \fontsize{10}{10} \selectfont $dp_date$}
  %\vspace*{1.5in}
  %{\color{white} \fontfamily{phv} \fontsize{38}{38} \selectfont CFPB Data Point:\\[25pt]
  %$dp_title$\\[40pt]
  %\fontsize{14}{14} \selectfont The CFPB Office of Research\par}

  % This is the new cover page
  \thispagestyle{fancy}
  \fancyhead[L]{\fontfamily{phv} \fontsize{10}{10} \selectfont $dp_date$}
  \ThisLLCornerWallPaper{1}{cfpb_report_bottom.png}
  \ThisCenterWallPaper{0.35}{cfpblogo_wide}
  \vspace*{1.5in}
  {\color{black} \fontfamily{phv} \fontsize{10}{10} \selectfont QUARTERLY CONSUMER CREDIT TRENDS\\[25pt]
  \\[20pt]
  \fontsize{38}{38} \selectfont $dp_title$\par}
\end{titlepage}

% Author Page
\raggedright
\fancyhead[L]{}  % This keeps the date on the cover page from repeating on other pages
\fancyfoot[R]{}
\fancyfoot[L]{\fontfamily{phv} \fontsize{8}{8} \selectfont \thepage\hspace{20pt}QUARTERLY CONSUMER CREDIT TRENDS: \uppercase{$dp_title$}}
\begin{itemize}     % [label=\FilledSmallSquare]
 \renewcommand{\labelitemi}{\scriptsize\FilledSmallSquare}
  $for(dp_author)$
    \item $dp_author$
  $endfor$
\end{itemize}
This is part of a series of quarterly reports of consumer credit trends produced by the Consumer Financial Protection Bureau using a longitudinal, nationally-\allowbreak representative sample of approximately five million de-identified credit records from one of the three nationwide credit reporting agencies.
\newpage

% This is the table of contents which will not appear in a qCCT
%\vspace*{1.5in}
%\tableofcontents
%\thispagestyle{fancy}
% \clearpage
% {
%   \pagestyle{empty}
%   \fancyfoot[L]{\fontfamily{phv} \fontsize{8}{8} \selectfont \thepage\hspace{20pt}CFPB DATA POINT: \uppercase{$dp_title$}}
% \vspace*{1.5in}
% \tableofcontents
% \thispagestyle{empty}
% }

% \newpage

% Body of the report
$body$

$if(bibliography)$
\newpage
\printbibliography[heading=bibintoc,title={References}]
$endif$

%\bibliography{$bibliography$}

%$if(natbib)$
%$if(biblio-files)$
%$if(biblio-title)$
%$if(book-class)$
%\renewcommand\bibname{$biblio-title$}
%$else$
%\renewcommand\refname{$biblio-title$}
%$endif$
%$endif$
%\newpage
%\vspace*{1.5in}
%{\fontfamily{phv} \fontsize{32}{32} \selectfont References}
%\bibliography{$biblio-files$}
%$endif$
%$endif$

%$if(biblatex)$
%\newpage
%\vspace*{1.5in}
%{\fontfamily{phv} \fontsize{32}{32} \selectfont References}
%\printbibliography
%$if(biblio-title)$[title=$biblio-title$]$endif$
%$endif$

\end{document}
